\chapter{规划}
\section{最短路径算法}
\subsection{Floyd算法}
Floyd算法采用动态规划思想,$f[k][i][j]$表示$i$和$j$之间可以通过编号为$1 \dots k$的节点的最短路径。初值$f[0][i][j]$为原图的邻接矩阵。

则$f[k][i][j]$可以从$f[k-1][i][j]$转移来,表示$i$到$j$不经过$k$这个节点。也可以从$f[k-1][i][k]+f[k-1][k][j]$转移过来,表示经过$k$这个点。意思即
\begin{equation*}
	f[k][i][j] = \rm{min}(f[k-1][i][j], f[k-1][i][k]+f[k-1][k][j])
\end{equation*}

核心代码为
\begin{lstlisting}[language=c,numbers=left,firstnumber = 1,numberstyle=\tiny,breaklines = true,keywordstyle=\color{blue!70},commentstyle=\color{red!50!green!50!blue!50},frame=shadowbox, rulesepcolor=\color{red!20!green!20!blue!20}]
	for(k=1;k<=n;k++)
        for(i=1;i<=n;i++)
            for(j=1;j<=n;j++)
                if(e[i][j]>e[i][k]+e[k][j])
                     e[i][j]=e[i][k]+e[k][j];
\end{lstlisting}

Floyd算法能够求解有向图中任意两个点之间的最短路径,且只能在不存在负权环的情况下使用,时间复杂度$O(N^3)$。

\subsection{Dijkstra算法}
Dijikstra算法采用贪心思想,维护两个点集$A$,$B$。$A$点集代表已经求出源点到该点的最短路的点的集合,$B$代表未求出源点到该点的最短路径的点的集合。维护一个向量$d$,$d[i]$代表源点到点i的最短路径长度。不断进行以下操作:找出点集$B$中$d[i]$最小的点,将这个点加入点集A中,然后用这个点到其邻接点的距离来更新向量$d$,直到点集B为空。

\href{http://wiki.jikexueyuan.com/project/easy-learn-algorithm/dijkstra.html}{\texttt{具体步骤}}为:
\begin{enumerate}
\item 将所有的顶点分为两部分:已知最短路程的顶点集合 P 和未知最短路径的顶点集合 Q。最开始,已知最短路径的顶点集合 P 中只有源点一个顶点。我们这里用一个 book[ i ]数组来记录哪些点在集合 P 中。例如对于某个顶点 i,如果 book[ i ]为 1 则表示这个顶点在集合 P 中,如果 book[ i ]为 0 则表示这个顶点在集合 Q 中。
\item 设置源点 s 到自己的最短路径为 0 即 dis=0。若存在源点有能直接到达的顶点 i,则把 dis[ i ]设为 e[s][ i ]。同时把所有其它(源点不能直接到达的)顶点的最短路径为设为 ∞。
\item 在集合 Q 的所有顶点中选择一个离源点 s 最近的顶点 u(即 dis[u]最小)加入到集合 P。并考察所有以点 u 为起点的边,对每一条边进行松弛操作。例如存在一条从 u 到 v 的边,那么可以通过将边 u->v 添加到尾部来拓展一条从 s 到 v 的路径,这条路径的长度是 dis[u]+e[u][v]。如果这个值比目前已知的 dis[v]的值要小,我们可以用新值来替代当前 dis[v]中的值。
\item 重复第 3 步,如果集合 Q 为空,算法结束。最终 dis 数组中的值就是源点到所有顶点的最短路径。
\end{enumerate}

Dijkstra算法能够求解图中单源点到其他所有点的最短路径。时间复杂度为$O(N^2)$

\subsection{Bellman-Ford算法}
\href{http://www.cnblogs.com/gaochundong/p/bellman_ford_algorithm.html}{\texttt{具体步骤}}为:
\begin{enumerate}
	\item 创建源顶点 v 到图中所有顶点的距离的集合 distSet,为图中的所有顶点指定一个距离值,初始均为 Infinite,源顶点距离为 0;
	\item 计算最短路径,执行V-1次遍历: 对于图中的每条边, 如果起点 u 的距离 d 加上边的权值 w 小于终点 v 的距离 d,则更新终点 v 的距离值 d;
	\item 检测图中是否有负权边形成了环: 遍历图中的所有边,计算 u 至 v 的距离,如果对于 v 存在更小的距离,则说明存在环;
\end{enumerate}

Bellman-Ford算法也能解决单源最短路问题,且对边的情况没有要求,不仅可以处理负权边,还能处理负环。时间复杂度为$O(V\times E)$

\subsection{A*算法}
个人理解A*算法只是针对搜索方向的一种启发式处理方法,在寻找最短路径的搜索过程中,通常都是以Dijkstra算法为基础进行的。具体的伪代码为:
\begin{lstlisting}[language=c,numbers=left,firstnumber = 1,numberstyle=\tiny,breaklines = true,keywordstyle=\color{blue!70},commentstyle=\color{red!50!green!50!blue!50},frame=shadowbox, rulesepcolor=\color{red!20!green!20!blue!20}]
function A*(start, goal)
    // The set of nodes already evaluated
    closedSet := {}

    // The set of currently discovered nodes that are not evaluated yet.
    // Initially, only the start node is known.
    openSet := {start}

    // For each node, which node it can most efficiently be reached from.
    // If a node can be reached from many nodes, cameFrom will eventually contain the
    // most efficient previous step.
    cameFrom := an empty map

    // For each node, the cost of getting from the start node to that node.
    gScore := map with default value of Infinity

    // The cost of going from start to start is zero.
    gScore[start] := 0

    // For each node, the total cost of getting from the start node to the goal
    // by passing by that node. That value is partly known, partly heuristic.
    fScore := map with default value of Infinity

    // For the first node, that value is completely heuristic.
    fScore[start] := heuristic_cost_estimate(start, goal)

    while openSet is not empty
        current := the node in openSet having the lowest fScore[] value
        if current = goal
            return reconstruct_path(cameFrom, current)

        openSet.Remove(current)
        closedSet.Add(current)

        for each neighbor of current
            if neighbor in closedSet
                continue		// Ignore the neighbor which is already evaluated.

            if neighbor not in openSet	// Discover a new node
                openSet.Add(neighbor)
            
            // The distance from start to a neighbor
            //the "dist_between" function may vary as per the solution requirements.
            tentative_gScore := gScore[current] + dist_between(current, neighbor)
            if tentative_gScore >= gScore[neighbor]
                continue		// This is not a better path.

            // This path is the best until now. Record it!
            cameFrom[neighbor] := current
            gScore[neighbor] := tentative_gScore
            fScore[neighbor] := gScore[neighbor] + heuristic_cost_estimate(neighbor, goal) 

    return failure

function reconstruct_path(cameFrom, current)
    total_path := [current]
    while current in cameFrom.Keys:
        current := cameFrom[current]
        total_path.append(current)
    return total_path
\end{lstlisting}
上述伪代码可以用于移动机器人路径搜索。其中closedSet表示已经搜索过的位置,openSet表示待搜索的位置。从openSet中按照fScore的值找到最近的待搜索点,从openSet中移入closedSet中。更新该点相邻可达点的fScore值和cameFrom表,同时将相邻点(不在closedSet)加入openSet中。不断重复直到搜索到达目标点,然后根据cameFrom表逆推得到最短路径。

与Dijkstra算法相比,A*算法最核心的地方就是对距离函数的取值:
\begin{equation*}
	f(n) = g(n) + h(n)
\end{equation*}
其中,$g(n)$就是Dijkstra算法中计算出的从源点到当前点的距离,$h(n)$则表示当前点到目标点的距离估计。利用该估计值启发搜索方向。这是我用Matlab编写的\href{/attachment/myastar.m}{\texttt{A*代码}}。为了便于Matlab搜索和定位,代码中将二维坐标转换为了一维坐标处理。

\section{路径搜索算法}
\subsection{PRM算法}
概率路图(Probabilistic RoadMap, PRM)算法。基本步骤分为两步:先对地图进行预处理建图,生成可行路径无向图,然后查询源点、终点在图上对应的最短距离。

\noindent {\hei $\blacksquare$ PRM算法预处理步骤}

\begin{enumerate}
\item 地图随机采样,生成中间路径点。
\item 按照不同的方法,在路径点之间生成路径
\item 对路径进行碰撞检测。
\end{enumerate}

在PRM算法中,常规方式是逐点采样,然后按照半径距离$r$连接到已生成的图中;一种简化方法(sPRM)是一次生成N个采样点,然后再逐点生成路径。此外,对最近若干点的挑选方式也有不同处理。常规是按照给定的半径球内寻点。k-Nearest (s)PRM 始终搜寻最近的k个点生成路径。Bounded-degree (s)PRM 则是对常规方式的球内点添加了k个最近上限。


\subsection{RRT算法}
快速扩展随机树(Rapidly-exploring Random Trees, RRT)算法。本节主要参考\href{http://rkala.in/codes/RRT.zip}{\texttt{Rahul Kala}}的代码以及\href{http://www.cnblogs.com/21207-iHome/p/7210543.html}{\texttt{这篇博客}}。

\noindent {\hei $\blacksquare$ RRT算法步骤}

\begin{enumerate}
\item 初始化随机树,确定初始点、目标点,地图尺寸、地图障碍物位置等
\item 随机采样。按照一定的概率在地图中挑选随机点或是用目标点作为当前采样点$q_\rm{rand}$
\item 选取最近节点。在当前随机树中找出与采样点最近的节点,作为扩展节点$q_\rm{nearest}$
\item 随机树扩展。沿$q_\rm{nearest}$和$q_\rm{rand}$方向,按照一定步长生成新节点$q_\rm{new}$
\item 碰撞检测。若$\overrightarrow{q_\rm{nearest} q_\rm{new}}$上的点都与障碍物无碰撞,则将$q_\rm{new}$加入随机树中。
\item 若$q_\rm{new}$距离目标点足够近则停止搜索,否则跳转步骤2。
\end{enumerate}

在实际使用是,还需要考虑:如何进行随机采样?如何定义“最近”以及如何快速搜索到最近节点(Kd-Tree)?如何进行扩展等。博客中举了一个车型机器人的例子,显然车的前后移动要比旋转、航向移动要方便,对应与更“近”。

\noindent {\hei $\blacksquare$ Bidirectional RRT/RRT Connect}

单源点开始搜索的RRT算法在搜索效率上仍显的较慢,因此有人提出如下的双向RRT搜索算法。
\begin{figure}[htbp]
	\figskip 
	\centering
	\includegraphics[width = 0.85\textwidth,trim = 0 -0 0 -0,clip]{rrt-connect.png}	  
	\caption{\label{fig: rrt-connect} RRT-Connect算法步骤}
\end{figure}
基本思想可以理解为:先对源点做一次RRT搜索,然后用扩展点对终点做一次RRT搜索,然后用源点RRT的扩展点与终点RRT的扩展点再做连续RRT拓展,直到遇到障碍物。然后重复上述步骤。为了平衡两棵随机树的节点,每一步搜索中的第一次的RRT可以对节点数多的进行。

\subsection{RRT*算法}
最优RRT(RRT*)算法,主要参考\href{/attachment/Sampling-based Algorithms for Optimal Motion Planning10.1.1.419.5503.pdf}{\texttt{这篇论文}}。简单来说,RRT*算法的“最优”性体现在对随机树的动态优化上,基本思想为:

\begin{enumerate}
    \item 随机树扩展。对于待扩展的新节点$q_\rm{new}$,不是简单的生成从$q_\rm{nearest}$到$q_\rm{new}$的路径,而是在$q_\rm{new}$的一个邻域内搜索所有节点$q_\rm{near}$,使得源点经过$q_\rm{near}$到$q_\rm{new}$的路径最短,这时再添加$q_\rm{near}$到$q_\rm{new}$路径。
    \item 随机树修正。对$q_\rm{new}$邻域内所有节点$q_\rm{near}$再做一次搜索,若源点经过新节点$q_\rm{new}$到$q_\rm{near}$的距离更短,则变更$q_\rm{near}$的父节点为$q_\rm{new}$,且替换原路径。
\end{enumerate}

\begin{figure}[htbp]
	\figskip 
	\centering
	\includegraphics[width = 0.9\textwidth,trim = 0 -0 0 -0,clip]{RRT*.png}	  
	\caption{\label{fig: RRT*} RRT*算法伪代码}
\end{figure}

\noindent {\hei $\blacksquare$ 疑问}

RRT*算法在每一步搜索中都需要寻找$q_\rm{new}$的邻域节点,这个搜索的代价是不是太大?如果按照常规RRT算法生成节点和路径,最后再做一次全局的最短路径搜索,性能对比如何?

\section{路径曲线生成算法}
\subsection{Dubins path}
对于平面路径生成,可以用$(x,y,\psi)'$表示状态,其中$x,y$为轨迹点的平面坐标位置,$\psi$表示轨迹点运动的轨迹偏角(对无人机,假设无侧滑角飞行)。Dubins曲线采用圆弧和直线连接源点和终点,对于包含直线的Dubins曲线,按照源点转弯方向和终点转弯方向,可以分为RSR,RSL,LSR,LSL共四种曲线形式。

\begin{figure}[htbp]
	\figskip 
	\centering
	\includegraphics[width = 0.75\textwidth,trim = 0 -0 0 -0,clip]{Dubins.png}	  
	\caption{\label{fig: Dubins} 四种Dubins曲线形式}
\end{figure}

具体算法思想可以参考\href{/attachment/Dubins.pdf}{\texttt{这篇论文}},具体的MATLAB实现可以看
\href{/attachment/dubins.m}{\texttt{这里}},主要注意下坐标系(x轴指向北)与论文中不一样,实现过程比较简单,都是几何运算。

基本思路都是:
\begin{enumerate}
	\item 按照路径形式确定源点、终点的转弯圆心
	\item 按照路径形式确定两圆的切线方向、航迹偏角
	\item 计算圆上切点
	\item 计算转弯旋转角度的变化增量
\end{enumerate}

关于如何根据条件状态直接选择路程最短的形式,可以参考王师姐论文。

\subsection{Bezier曲线 和 B-spline曲线}
\noindent {\hei $\blacksquare$ 贝塞尔曲线}

参考\href{https://www.cnblogs.com/hnfxs/p/3148483.html}{\texttt{这篇博客}}。给定源点$P_0$和终点$P_1$,那么一阶Bezier曲线就是这两个点的连线,连线上的点可以描述为:
\begin{equation*}
    P = (1-t)P_0 + tP_1
\end{equation*}
其中,$t: 0 \Rightarrow 1$。这个形式相当于一个一阶插值。类似的,二阶Bezier曲线中,一共需要3个控制点$P_0$,$P_1$,$P_2$,先计算$P_0$和$P_1$的曲线点$P_0^1$,再计算$P_1$和$P_2$的曲线点$P_1^1$,最后计算$P_0^1$和$P_1^1$的曲线点就是二阶曲线上的点。

\begin{align*}
    P &= (1-t)\left[(1-t)P_0 + tP_1\right] + t\left[ (1-t)P_1 + tP_2\right] \\
      &= (1-t)^2 P_0 + 2t(1-t)P_1 + t^2 tP_2
\end{align*}

贝塞尔曲线存在的问题:
\begin{enumerate}
	\item 确定了多边形的顶点数(m个),也就决定了所定义的Bezier曲线的阶次(m-1次),不够灵活。
    \item 当顶点数(m)较大时,曲线的阶次将比较高。此时,多边形对曲线形状的控制将明显减弱。    
    \item Bezier的调和函数的值,在开区间(0,1)内均不为0。因此,所定义的曲线在(0<t<1)的区间内的任何一点均要受到全部顶点的影响,即改变其中任一个顶点的位置,都将对整条曲线产生影响,因此对曲线进行局部修改是不可能的。
\end{enumerate}

\noindent {\hei $\blacksquare$ B样条曲线}

可以参考\href{https://blog.csdn.net/Mr_Grit/article/details/45603627}{\texttt{这篇博客}}。讲的比较清楚,而且代码也可以运行通过。

一般使用其中的准均匀B样条曲线,需要特别注意其中求取节点的地方,包括数量及其分布。节点可以理解为在原控制点上的扩充,用于计算基函数。控制点就是给定的控制多边形的定点。




\subsection{其他形式曲线}
\noindent {\hei $\blacksquare$ Reeds-Shepp曲线}

与Dubins曲线相比,RS曲线中机器人既能够前向运动,也能够后向运动(类似具有倒车的性质)。共计9类46种曲线直线组合结果。

\noindent {\hei $\blacksquare$ Balkcom-Mason曲线}

与RS曲线相比,Balkcom-Mason曲线中机器人可以绕中心旋转(类似双轮机器人差动转向的性质)。

\noindent {\hei $\blacksquare$ Pythagorean Hodograph曲线}

PH曲线是一种具有有理特性的参数化多项式曲线。具体计算可以根据师姐论文中的方法,在复平面内进行计算。

