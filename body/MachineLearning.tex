\chapter{机器学习}
这一部分主要是对JDAI中级培训的内容总结,对于课程中重要但又未详细讲解的地方,也会添加一些其他教材中的内容作为补充。

\section{准备知识}
\subsection{概述}
机器学习的基本过程:

1. 监督学习(有已知样本):信息获取与预处理-->特征提取与特征选择-->分类器设计(训练)-->分类决策(识别)

2. 非监督学习(无已知样本):信息获取与预处理-->特征提取与特征选择-->聚类(自学习)-->结果解释

机器学习技术主要包括:

1. 分类:决策树、K近邻、逻辑回归、支持向量机、神经网络等。

2. 聚类:C均值、模糊C均值、改进的模糊C均值等。

机器学习的一些基本名词:
\begin{enumerate}
\item 特征(Feature):样本的任何可区分且可观测的方面。包括定量、定性特征,但通常都转化为定量特征。特征向量、特征空间。
\item 分类器:能够将每一个样本都分到某个类别中(或者拒绝)的算法。
\end{enumerate}

\subsection{统计学习三要素}
统计学习方法=模型+策略+算法。以下对监督学习进行阐述。

{\hei{模型}}就是所要学习到的条件概率分布或是决策函数。假设空间(hypothesis space)就是所有可能模型组成的集合。
决策函数表示的模型为非概率模型,条件概率分布表示的模型为概率模型。

{\hei{策略}}可以理解为确定优化目标函数,即按照什么样的准则来学习/选择最优模型。
损失函数度量模型一次预测的好坏,风险函数度量平均意义下的模型预测好坏。学习的目标是选择期望风险最小的模型。

训练数据集上的平均损失为经验风险。经验风险最小化、结构风险最小化。结构风险最小化等价于正则化。

{\color{red} 损失函数、代价函数、目标函数的区别}
\begin{enumerate}
\item 损失函数(loss fun.),也可以理解为误差函数(error fun.),针对单个样本点。
\item 代价函数(cost fun.),针对整个样本集/训练集而言。可以认为是所有样本点损失函数的平均。
\item 目标函数(object fun.),一般是代价函数+正则项。    
\end{enumerate}

{\hei{算法}}是指学习模型的具体计算方法。


\subsection{模型评估}
模型评估的必要性?

{\hei{$\blacksquare$ 过拟合问题}}

Mitchell在Machine Learning中给了一个定义:Given a hypothesis space $H$, a hypothesis $h \in H$ 
is said to overfit the training data if there exists some alternative hypothesis $h' \in H$,
such that $h$ has smaller error than $h'$ over the training examples, but $h'$ has a smaller
error than $h$ over the entire distribution of instances.

在这个定义里并没有严格证明一定存在$h'$,所以可以理解过拟合也是一个相对而言的问题。
简单来说,过拟合指所选取的模型在已知数据集上的预测效果很好,但对未知数据预测效果很差的现象。
过拟合本质是一味追求在已知数据上的预测能力,选取的模型复杂度往往比真实模型更高,使得将样本数据本身所包含的噪声也作为真实数据进行学习。

什么是模型复杂度?
如何解决过拟合问题?

{\hei{$\blacksquare$ 混淆矩阵}}

针对实际问题时,主要关注混淆矩阵中非对角线上的元素,也就是分类器判断错误的样本。

多分类问题也可以生成混淆矩阵,只不过不再是$2 \times 2$的矩阵(类别True、False),而是类别数的$n \times n$的矩阵。

{\hei{$\blacksquare$ ROC曲线}}

\subsection{交叉熵代价函数}
{\hei{$\blacksquare$ 信息熵}}: 消除不确定性所需要的信息量的度量。

令事件x的信息量表示为$I(x)$,假设信息量与概率有关。
考虑两个独立不相关事件x和y,显然有$I(x,y) = I(x)+I(y)$,而$p(x,y)=p(x)*p(y)$。
所以可以定义信息量为:
\begin{equation*}
    I(x) = -\log p(x)
\end{equation*}
取负号是为了保证信息量为正。对数函数可以任意选取,信息论中以2为底,机器学习通常以e为底。
对于随机变量X,信息熵为:
\begin{equation*}
    H(X) = \mathbb{E}[-\log p(X)] = - \sum_i p(x_i) \log p(x_i)
\end{equation*}

{\hei{$\blacksquare$ 相对熵(KL散度)}}:描述两个概率分布的非对称性度量。定义为
\begin{equation*}
    D_{KL}(p||q) = -\sum_i p(x_i) \log \frac{q(x_i)}{p(x_i)}
\end{equation*}
{\color{red}尽管从直觉上KL散度是个度量或距离函数, 但是它实际上并不是一个真正的度量或距離。因為KL散度不具有对称性:从分布P到Q的距离通常并不等于从Q到P的距离。
$D_{KL}(p||q) \neq D_{KL}(q||p)$} 

通常$p(x)$表示真实(样本)概率分布,$q(x)$表示模型概率分布。显然可以使用相对熵作为代价函数,使得模型生成的概率分布逼近真实样本的概率分布。
所以进一步,对$D_{KL}(p||q)$变换可得
\begin{align*}
    D_{KL}(p||q) &= -\sum_i p(x_i) \log q(x_i) + \sum_i p(x_i) \log p(x_i) \\
    &= H(p,q) - H(p)
\end{align*}
其中$H(p,q) = -\sum_i p(x_i) \log q(x_i)$即表示交叉熵。
由于真实(样本)概率分布是固定的,所以交叉熵也可以作为代价函数,计算更简便。

对于二分类问题,单个样本点上的损失函数可以表示为二元交叉熵的形式:
\begin{equation*}
    L = -[y\log \hat{y} + (1-y)\log (1-\hat{y})]
\end{equation*}

\subsection{对数损失函数}
对数损失函数为
\begin{equation*}
    L(Y, P(Y|X)) = -\log P(Y|X)
\end{equation*}



\section{Logistic Regression, LR}
二分类LR模型为:
\begin{gather*}
    P(y=1|x) = \frac{e^{w\cdot x + b}}{1 + e^{w\cdot x + b}}\\
    P(y=0|x) = \frac{1}{1 + e^{w\cdot x + b}}
\end{gather*}
其中$y\in{0,1}$为输出。

假设一个事件的发生概率为$p$,定义事件发生的几率(odds)为发生概率与不发生概率的比值,则对数几率(log odds)或logit函数:
\begin{equation*}
    \text{logit} = \log \frac{p}{1-p}
\end{equation*}
那么对LR而言,有:
\begin{equation*}
    \text{logit}(x) = \log\frac{P(y=1|x)}{1-P({y=1|x})} = w\cdot x + b
\end{equation*}
也就是说线性运算的结果对应表示该事件$A(y=1|x)$的对数几率。

可以通过比较$P(y=1|x)$和$P(y=0|x)$的大小,将$x$进行分类。但logistic决策的本质是根据logit(x)判断:
若$\text{logit}(x) > 0$,则$x\in y_1$;若$\text{logit}(x) < 0$,则$x\in y_0$

可以将二分类LR模型推广到多分类LR模型,其实就是Softmax回归:
% \begin{gather*}
%     P(y=k|x) = \frac{\text{exp}(w_k)\cdot x}{1 + \sum_{k=1}^{K-1} \text{exp}(w_k \cdot x)},\, k=1,2,\cdots,K-1 \\
%     P(y=K|x) = \frac{1}{1 + \sum_{k=1}^{K-1} \text{exp}(w_k \cdot x)}
% \end{gather*}
\begin{equation*}
    P(y=k|x) = \frac{\text{exp}(w_k\cdot x + b_k)}{\sum_{j=1}^K \text{exp}(w_j \cdot x + b_j)}
\end{equation*}

\subsection{Sigmoid函数与Softmax函数的区别}
Sigmoid函数形式为:
\begin{equation*}
    \sigma(x) = \frac{1}{1 + e^{-x}} = \frac{e^{x}}{1 + e^{x}}
\end{equation*}

Softmax函数形式为:
\begin{equation*}
    \textbf{Softmax}(x_j) = \frac{e^{x_j}}{\sum_{k=1}^K e^{x_k}},\, j=1,2,\cdots,K
\end{equation*}

Sigmoid函数只考虑单变量,将其映射到(0,1)区间内,即使对向量做运算,向量每个元素间的结果无联系。
Softmax函数则考虑了向量中的每一个元素,虽然每个元素的值也是映射到(0,1)区间内,但各元素之和为1。

对于二值分类问题,Sigmoid函数和Softmax函数都输出两个值,假设输出[0, 1]代表“是”,输出[1, 0]代表“否”。\\
那么Softmax可能输出[0.3, 0.7],代表算法认为“是”的概率是0.7,“否”的概率是0.3,相加为1。\\
Sigmoid的输出可能是[0.4, 0.8],它们相加不为1。
解释来说就是Sigmoid认为输出第一位为1的概率是0.4,第一位不为1的概率是0.6(1-p),第二位为1的概率是0.8,第二位不为1的概率是0.2。

\subsection{Sigmoid函数多分类与Softmax函数多分类的区别}
按照是否属于该类别的思路,可以设计多个使用Sigmoid函数的LR分类器。但是这样对每个分类器训练的时候,是/不是该类别的数据样本量会出现偏差。
输出的结果向量中,每一个元素表示是/不是情况下,是该类别的概率,每个元素间不相关。

而使用Softmax多分类,则输出结果表示每一个类别的可能性。

\subsection{二分类逻辑回归求解}
二分类逻辑回归模型可以统一表示为:
\begin{equation*}
    P(y|x;\theta) = (\sigma_\theta(x))^y \cdot (1-\sigma_\theta(x))^{(1-y)}
\end{equation*}
其中,$\sigma_\theta(x) = \sigma(\theta^T x)$。
二分类逻辑回归就是在样本$X$上找到最大可能的概率分布$P(Y|X;\theta)$,显然是一个求解极大似然的问题。
似然函数:
\begin{align*}
    L(\theta) &= P(Y|X;\theta) = \prod_i P(y_i|x_i;\theta) 
\end{align*}
取对数并取反,即采用对数损失函数(交叉熵代价函数)
\begin{align*}
    L = -\log L(\theta) &= -\log P(Y|X;\theta) = -\sum_i P(y_i|x_i;\theta) \\ 
    &= - \sum_i \left[ y_i \log (\sigma_\theta(x_i)) + (1-y_i) \log (1-\sigma_\theta(x_i))\right]
\end{align*}

由于$\nabla_x \sigma(x) = \sigma(x)(1-\sigma(x))$,所以对$L$求梯度可得:
\begin{equation*}
    \nabla_\theta L = -\sum_i (y_i - \sigma_\theta(x_i))x_i
\end{equation*}

{\hei{$\bigstar$ 为什么逻辑回归要使用交叉熵代价函数?}}

1. 为什么可以使用交叉熵代价函数:目的是为了让样本预测的概率可能性最大,
并且训练求解参数的速度是比较快的,而且更新速度只和x,y有关,比较的稳定。

2. 为什么不用平方损失函数:梯度更新速度和 sigmod 函数的梯度相关,训练速度会非常慢且有梯度消失的可能。
而且平方损失会导致损失函数是 theta 的非凸函数,不利于求解,因为非凸函数存在很多局部最优解。

使用交叉熵代价(对数损失)函数的本质目的在于极大似然的原理。即使激活函数使用ReLU,仍然可以使用。

\subsection{softmax多分类求解}
采用softmax的多分类逻辑回归模型可以统一表示为:
\begin{equation*}
    P(y|x;\theta) = \sum_{i=1}^k \mathbb{1}\{ y_j=i \} \frac{e^{\theta_i^T x}}{\sum_k e^{\theta_k^T x}}
\end{equation*}
同样按照求解极大似然估计问题,取对数损失函数为:
\begin{align*}
    L &= -\log P(Y|X;\theta) = -\sum_j P(y_j|x_j;\theta) \\
    &= -\sum_{j=1}^{m} \sum_{i=1}^k \mathbb{1}\{ y_j=i \} \log \left[\frac{e^{\theta_i^T x_j}}{\sum_k e^{\theta_k^T x_j}} \right] \\
    &= -\sum_{j=1}^{m} \sum_{i=1}^k \mathbb{1}\{ y_j=i \} \left( \theta_i^T x_j - \log \sum_k e^{\theta_k^T x_j} \right)\\
\end{align*}
求梯度可的:
\begin{align*}
    \Q{L(\theta)}{\theta_i} &= -\sum_{j=1}^{m} \mathbb{1}\{ y_j=i \} \left(x_j - \Q{\log \sum_k e^{\theta_k^T x_j}}{\theta_i} \right)\\
    &= -\sum_{j=1}^{m} \mathbb{1}\{ y_j=i \} \left(x_j - x_j \frac{e^{\theta_i^T x_j}}{\sum_k e^{\theta_k^T x_j}} \right)\\
    &= -\sum_{j=1}^{m} x_j \left[ \mathbb{1}\{ y_j=i \} - P(y_j=i|x_j;\theta) \right]
\end{align*}