\chapter{代码总结}
\section{代码}
\subsection{const 关键字的使用}
\href{https://blog.csdn.net/lf1570180470/article/details/56677748}{参考1}
\href{https://www.cnblogs.com/azbane/p/7266747.html}{参考2}

1. const修饰一个变量,表明其为一个常量,初始化赋值后不能被修改。

2. const修饰指针,可以定义为一个指针常量(int * const p),其只能指向初始化的地址; 可以定义为一个常量指针(const int * p), 其只能指向常量; 最后可以定义一个指向常量的指针常量(const int * const p)

3. const修饰函数的传入参数,表明其在函数体内部不允许被修改。一般只对非内部类型,使用常量引用的形式const \&

4. const修饰类成员函数返回值,表明其不能作为左值使用

5. const修饰类成员函数,表明该函数不允许修改类的成员变量。任何不会修改数据成员的函数都应该声明为const类型: int fun(int a) const;

\subsection{用标准库构造堆}
\href{http://www.cplusplus.com/reference/algorithm/make_heap/?kw=make_heap}{参考}

1. 对于一个vector,可以使用std::make_heap生成一个堆,默认是大顶堆(less<int>()),如果使用小顶堆,比较器使用greater<int>()

2. std::pop_heap,将堆顶元素移到vector的末尾,同时进行调整。小顶堆使用greater<int>()

3. std::push_heap,将vector末尾的元素视为新插入的元素,并进行调整。小顶堆使用greater<int>()

4. 调整后的结果并不是对vector进行排序。std::sort_heap则是进行排序。