\chapter{C++}
\section{基础算法}
排序算法的稳定性:排序前后相同元素的相对位置不变,则称排序算法是稳定的;否则排序算法是不稳定的
\subsection{快速排序}
平均时间复杂度$O(n\log n)$,最快情况为$O(n^2)$

\begin{lstlisting}[language=c++,numbers=left,firstnumber = 0,numberstyle=\tiny,breaklines = true,keywordstyle=\color{blue!70},commentstyle=\color{red!50!green!50!blue!50},frame=shadowbox, rulesepcolor=\color{red!20!green!20!blue!20}]
	
template<typename T>
void quick_sort(vector<T> &nums, int left, int right){
    if (left >= right) return;
    int low = left;
    int high = right;
    T val = nums[left];
    while(left < right) {
        while(left<right && nums[right] >= val) right--;
        nums[left] = nums[right];
        while(left<right && nums[left] <= val) left++;
        nums[right] = nums[left];
        }
    nums[left] = val;
    quick_sort(nums, low, left-1);
    quick_sort(nums, left+1, high); 
}
\end{lstlisting}

\subsection{堆排序}
平均时间复杂度$O(n\log n)$,最快情况为$O(n\log n)$

\begin{lstlisting}[language=c++,numbers=left,firstnumber = 0,numberstyle=\tiny,breaklines = true,keywordstyle=\color{blue!70},commentstyle=\color{red!50!green!50!blue!50},frame=shadowbox, rulesepcolor=\color{red!20!green!20!blue!20}]

template<typename T>
void heap_max(vector<T> &nums, int start, int end){
    int dad = start;
    int son = dad * 2 + 1;
    while(son <= end){
        if (son+1 <= end && nums[son] < nums[son+1]) son++;
        if (nums[dad] >= nums[son]) return;
        else {
            swap(nums[dad], nums[son]);
            dad = son;
            son = son * 2 + 1;
        }
    }
    return;
}

template<typename T>
void heap_sort(vector<T> &nums, int start, int end){
    for(int i = end; i >= 0; --i){
        heap_max(nums, i, end);
    }
    for(int i=0; i <= end; ++i){
        swap(nums[0], nums[end-i]);
        heap_max(nums, 0, end-1-i);
    }
}

\end{lstlisting}


\subsection{计数排序}
平均时间复杂度和最坏时间复杂度都是$O(n+k)$,稳定排序。
\begin{lstlisting}[language=c++,numbers=left,firstnumber = 0,numberstyle=\tiny,breaklines = true,keywordstyle=\color{blue!70},commentstyle=\color{red!50!green!50!blue!50},frame=shadowbox, rulesepcolor=\color{red!20!green!20!blue!20}]
void countsort(vector<int> &nums, int start, int end){
    int min = 0x7fffffff;
    int max = 0x80000000;
    for(auto n:nums){
        if (min>n) min=n;
        if (max<n) max=n;
    }
    vector<int> counts(max-min+1, 0);
    vector<int> ans(nums.size(),0);
    for(auto n:nums){
        counts[n-min]++;
    }
    for(int i=1; i<counts.size(); ++i) counts[i] += counts[i-1];
    for(int i=0; i<nums.size(); ++i) {
        ans[counts[nums[i]-min]-- -1] = nums[i]; //--对应于重复的数字
    }
    for(int i=0; i<nums.size(); ++i) nums[i] = ans[i];
}
\end{lstlisting}




\section{语言使用}
\subsection{某量声明与定义}
\begin{enumerate}
\item 声明规定了变量的类型和名字,需要使用extern关键字,且没有显式地初始化变量
\item 定义除了声明的作用外,还申请了存储空间,也可能会给变量赋予初始值
\end{enumerate}

\subsection{指针与引用}
表现出的本质区别:指针是一个对象,而引用不是对象,只是某个已存在对象的别名。
\begin{enumerate}
\item 定义引用的时候,必需初始化引用的对象。这时引用将会绑定到该对象。而指针可以指向空地址、野值。
\item 引用无法解绑定,也无法变更绑定的对象。指针能多次赋值、更换指向的对象。
\item 解指针操作后,得到指向的对象;而引用本身就是引用的对象。
\item sizeof引用得到的是所指向对象的大小,而sizeof指针是得到指针本身。(这里有数组指针的问题)
\item 指针和引用的自增自减运算意义不一样。
\end{enumerate}

但如果将引用视为一个自动包含解指针操作的指针常量,那么指针与引用就没有本质区别了。
\href{https://www.marvinle.cn/2018/12/10/pointer-reference/}{参考这里}:
引用只是c++语法糖,可以看作编译器自动完成取地址、解引用的常量指针

\subsection{参数缺省}
1. 只在声明中表明参数的缺省值

2. 从右向左增加缺省值

3. 使用时从左向右传入参数,无缺省值的参数必需指定实参。



\subsection{const 关键字的使用}
\href{https://blog.csdn.net/lf1570180470/article/details/56677748}{参考1}
\href{https://www.cnblogs.com/azbane/p/7266747.html}{参考2}

1. const修饰一个变量,表明其为一个常量,初始化赋值后不能被修改。

2. const修饰指针,可以定义为一个指针常量(int * const p),其只能指向初始化的地址; 可以定义为一个常量指针(const int * p), 其只能指向常量; 最后可以定义一个指向常量的指针常量(const int * const p)

3. const修饰函数的传入参数,表明其在函数体内部不允许被修改。一般只对非内部类型,使用常量引用的形式const \&

4. const修饰类成员函数返回值,表明其不能作为左值使用

5. const修饰类成员函数,表明该函数不允许修改类的成员变量。任何不会修改数据成员的函数都应该声明为const类型: int fun(int a) const;

\subsection{用标准库构造堆}
\href{http://www.cplusplus.com/reference/algorithm/make_heap/?kw=make_heap}{参考}

1. 对于一个vector,可以使用std::make_heap生成一个堆,默认是大顶堆(less<int>()),如果使用小顶堆,比较器使用greater<int>()

2. std::pop_heap,将堆顶元素移到vector的末尾,同时进行调整。小顶堆使用greater<int>()

3. std::push_heap,将vector末尾的元素视为新插入的元素,并进行调整。小顶堆使用greater<int>()

4. 调整后的结果并不是对vector进行排序。std::sort_heap则是进行排序。