\chapter{LeetCode}
\section{数论}
\subsection{Power of Three}

\noindent {\hei $\blacksquare$ \href{https://leetcode.com/problems/power-of-three/description/}{\texttt{题目描述}}}:

给一个整数,判断是否为3的整数幂。要求不使用循环/递归。

\noindent {\hei $\blacksquare$ 解答思路一}:

若整数$N$是3的整数幂,即$3^a = N$ 那么$a = \log_3 N$一定也是一个整数,使用换底公式为:
\begin{equation*}
    a = \frac{\log_{10} N}{\log_{10} 3} 
\end{equation*}
注,不能使用自然对数$e$或2为底进行计算,无法通过。
判断$a$是否为整数可以用\texttt{(a - (int)a == 0)}来判断。

\noindent {\hei $\blacksquare$ 解答思路二}:

可以用int范围内最大的一个3的整数幂来求余,若余数为0,表明$N$是3的整数幂。

\section{树}
\subsection{Recover Binary Search Tree}

\noindent {\hei $\blacksquare$ \href{https://leetcode.com/problems/recover-binary-search-tree/description/}{\texttt{题目描述}}}:

给一个二叉搜索树,其中有两个节点位置交换了,需要将其恢复。要求空间复杂度为O(N)。

\noindent {\hei $\blacksquare$ 解答思路}: \\
1. 先序遍历二叉搜索树时,遍历结果为一个递增序列。若出现某个元素大于后一个元素,则说明这两个元素的位置有误。\\
2. 要求只使用O(N)空间复杂度,所以不能用常规递归/迭代的方式进行遍历,需要使用Morris traversal的遍历方式。具体算法为:\\

A...如果左子树为空,打印当前节点,并将当前节点指向右子树。

B...如果左子树非空,迭代找到左子树中的先序遍历的最后一个节点。

.......a) 如果该节点的右子树指向当前节点,则将恢复右子树为NULL,当前节点指向右子树。

.......b) 如果该节点的右子树为空,则将右子树指向当前节点,打印当前节点,并将当前节点指向左子树。

C...迭代重复上述过程直到当前节点为空。

\noindent 3. 需要将上述算法中节点打印的顺序用一个指针记录。

\begin{lstlisting}[language=c++,numbers=left,firstnumber = 1,numberstyle=\tiny,breaklines = true,keywordstyle=\color{blue!70},commentstyle=\color{red!50!green!50!blue!50},frame=shadowbox, rulesepcolor=\color{red!20!green!20!blue!20}]
struct TreeNode {
    int val;
    TreeNode *left;
    TreeNode *right;
    TreeNode(int x) : val(x), left(NULL), right(NULL) {}
};

int MorrisTravelPreorder(TreeNode* root){
    TreeNode *cur = root;
    TreeNode *pre = NULL;
    while(cur){
        if (!cur->left) {
            cout << cur->val << endl;
            cur = cur->right;
        } else {
            pre = cur->left;
            while(pre->right && pre->right != cur){
                pre = pre->right;
            }
            if (!pre->right) {
                pre->right = cur;
                cur = cur->left;
            } else {
                pre->right = NULL;
                cout << cur->val << endl;
                cur = cur->right;
            }
        }
    }
}
\end{lstlisting}