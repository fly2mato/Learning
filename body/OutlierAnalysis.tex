\chapter{异常检测}
这一部分是对《Outlier Analysis》的学习总结。

\section{绪论}
异常检测的应用场景很广泛:入侵检测系统、信用卡欺诈、兴趣检测、医疗诊断等。
在这些应用中,数据往往具有一个“正常”模型,而偏离这些正常模型的数据被称为非正常。
在一些特定应用场景中,异常值(outliers)还包括序列异常、集体异常等。

异常检测算法的输出分为两类:Outlier Scores 和 Binary labels

建立一个判别异常值的充分条件往往是一件非常主观的事情,因为很多时候数据会混入{\hei 噪声}。
从正常数据向非正常数据变化的过程中,噪声会模糊二者的边界。本书中的outliers包含噪声、非正常数据。
从outlier scores的角度看,非正常数据的评分普遍比噪声数据高,但二者之间也没有明确的边界。

所以从噪声的角度出发,去噪、噪声检测算法所涉及的本质与异常检测类似。
而异常值与噪声的区别,则需要分析人员基于特定的兴趣点(异常范例)进行判别。