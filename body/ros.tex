\chapter{ROS}
\section{基础}
\subsection{创建工作空间}
\begin{lstlisting}[language=bash,numbers=left,firstnumber = 1,numberstyle=\tiny,breaklines = true,keywordstyle=\color{blue!70},commentstyle=\color{red!50!green!50!blue!50},frame=shadowbox, rulesepcolor=\color{red!20!green!20!blue!20}]
mkdir -p ~/workspace/src
cd ~/workspace
catkin_make
source devel/setup.bash
\end{lstlisting}

\subsection{创建一个pkg}
\begin{lstlisting}[language=bash,numbers=left,firstnumber = 1,numberstyle=\tiny,breaklines = true,keywordstyle=\color{blue!70},commentstyle=\color{red!50!green!50!blue!50},frame=shadowbox, rulesepcolor=\color{red!20!green!20!blue!20}]
cd ~/workspace/src
catkin_create_pkg pkgname std_msgs rospy roscpp
cd ~/workspace
catkin_make
\end{lstlisting}







\subsection{编译git上的pkg}
1. 创建一个空的文件夹作为工作空间目录

2. git clone 待使用的目标package

3. 在工作空间目录下,使用\texttt{catkin_make --source pkgname}进行编译

4. (似乎可以忽略)使用\texttt{catkin_make install --source pkgname}安装。

5. 使用\texttt{source devel/setup.bash}读取当前工作空间的环境变量。




\subsection{问题及解决方案}
编译报错:ModuleNotFoundError: No module named 'em'
使用 "~\$ pip install em"
编译继续报错:AttributeError: module 'em' has no attribute 'Interpreter'
解决:

pip uninstall em
pip install empy