% 页面设置
\usepackage{CJKnumb}
\usepackage[top = 2.54cm, bottom = 2.54cm, left = 2.5cm, right = 2.5cm]{geometry}     %%% 按照西工大要求进行修改
\usepackage{indentfirst}                         % 首行缩进宏包
\usepackage[sf]{titlesec}                        % 控制标题的宏包
\usepackage{titletoc}                            % 控制目录的宏包
\usepackage{fancyhdr}                            % 自定义页眉页脚
\usepackage{fancyref}                            % 引用链接属性
\usepackage[perpage,symbol*,marginal]{footmisc}            % 脚注控制
\usepackage{layouts}                             % 打印当前页面格式的宏包
\usepackage{paralist}                            % 一种换行不缩进的列表格式,asparaenum,inparaenum 等
\usepackage[shortlabels]{enumitem}               % 列表格式
\usepackage{fancyvrb}                            % 原样输出
\usepackage[amsmath,thmmarks,hyperref]{ntheorem} % 定理类环境宏包
\usepackage{type1cm}                             % 控制字体的大小
\usepackage{enumitem}
\usepackage[x11names]{xcolor} % 支持彩色  %放置在{pdfpages}之前,避免冲突
\usepackage{pdfpages}
% 表格处理
\usepackage{booktabs}   % 三线表
\usepackage{multirow}   % 表格多行处理
\usepackage{diagbox}    % 斜线表头
\usepackage{tabularx}   % 表格折行
% \usepackage{siunitx}    % 国际单位,小数点对齐

% 图形相关
\usepackage{graphicx}         % 请在引用图片时务必给出后缀名
% \usepackage[x11names]{xcolor} % 支持彩色 提前放置到{pdfpages}之前!!
% \usepackage[below]{placeins}  % 浮动图形控制宏包                            %%%待修改。
\usepackage[section]{placeins}  % 浮动图形控制宏包  \FloatBarrier          参考《LaTeX2e插图指南》P92。在每个section前都清理浮动图形。比below严格。
\usepackage{rotating}	      % 图形和表格的控制
\usepackage{picinpar}
\usepackage{setspace}         % 定制表格和图形的多行标题行距
\usepackage{subfigure}           % 插入子图形
\usepackage[subfigure]{ccaption} % 插图表格的双语标题

\usepackage{listings}         % 源代码展示
\lstset{%
  language=TeX,
  defaultdialect=empty,
  basicstyle=\ttfamily\small,
  backgroundcolor=\color{LightSteelBlue1},
  keywordstyle=\color{blue},
  showspaces=false,
  showstringspaces=false,
  showtabs=false,
  tabsize=2,breakatwhitespace=false,
  columns=flexible}

% 其他
\usepackage{calc}   % 在 tex 文件中具有一些计算功能,主要用在页面控制。
\usepackage[xetex,
bookmarksnumbered=true,
bookmarksopen=true,
colorlinks=true,
% pdfborder={0 0 1},
citecolor=blue,
linkcolor=blue,%magenta
anchorcolor=green,
urlcolor=magenta,
breaklinks=true,
CJKbookmarks=true,
]{hyperref}


\usepackage[numbers,sort&compress,square,super]{natbib} %参考文献
\usepackage{hypernat}
\usepackage{bibentry}

\usepackage{rotating}
\usepackage{lscape}

\usepackage{mfirstuc}
\usepackage{soul} %给文本加入高亮显示
\usepackage{mdwlist}
\usepackage{bm} %用\bm输入黑斜体
\usepackage{changepage} %最后版权页中需要使用,段落整体缩进 
\usepackage{cases} %为了使用{numbercases}环境
\usepackage{amsmath} %到底要不要加上?
% \usepackage{float} %为了插入图片时使用H模式,但是该模式较不方便。



\usepackage{flafter} %使图形在引述后出现
\usepackage{xpatch} %打补丁,修改chapter

\usepackage{remreset} %图片表格编号全局编排

\usepackage{underscore} 

\usepackage{tikz,mathpazo}
\usetikzlibrary{shapes.geometric, arrows}
\usetikzlibrary{calc}

\usepackage{ulem} %下划线、删除线
\usepackage{algorithm}
\usepackage{algorithmicx}
 