%%%%%%%%%%%%%%%%%%%%%%%%%%%%%%%%%%%%%%%%%%%%%%%%%%%%%%%%%%%%%%%%%%%%%%
% 页面设置
%%%%%%%%%%%%%%%%%%%%%%%%%%%%%%%%%%%%%%%%%%%%%%%%%%%%%%%%%%%%%%%%%%%%%%
% A4 纸张
%\setlength{\paperwidth}{21.0cm}
\setlength{\paperwidth}{23.0cm} 
\setlength{\paperheight}{29.7cm}
% 设置正文尺寸大小
% \setlength{\textwidth}{16.1cm}
\setlength{\textheight}{24.5cm}  %22.2    %如果注释掉,会多显示一行
% % 设置正文区在正中间
% \newlength \mymargin
% \setlength{\mymargin}{(\paperwidth-\textwidth)/2}
% \setlength{\oddsidemargin}{(\mymargin)-1in}
% \setlength{\evensidemargin}{(\mymargin)-1in}
% % 设置正文区偏移量,奇数页向右偏,偶数页向左偏
% \newlength \myshift
% \setlength{\myshift}{0.35cm}	% 双面打印的奇偶页偏移值,可根据需要修改,建议小于 0.5cm
% \addtolength{\oddsidemargin}{\myshift}
% \addtolength{\evensidemargin}{-\myshift}
% % 页眉页脚相关距离设置
% \setlength{\topmargin}{-0.05cm}
% \setlength{\headheight}{0.50cm}
% \setlength{\headsep}{0.90cm}
% \setlength{\footskip}{1.47cm}
\setlength{\headheight}{15.7pt}

% 公式的精调
\allowdisplaybreaks[4]  % 可以让公式在排不下的时候分页排,这可避免页面有大段空白。

% 下面这组命令使浮动对象的缺省值稍微宽松一点,从而防止幅度
% 对象占据过多的文本页面,也可以防止在很大空白的浮动页上放置很小的图形。
\renewcommand{\topfraction}{0.9999999}
\renewcommand{\textfraction}{0.0000001}
\renewcommand{\floatpagefraction}{0.9999}

% \makeatletter %原
% \renewcommand\normalsize{%
%    \@setfontsize\normalsize\@xpt\@xiipt
%    \abovedisplayskip 3\p@ \@plus2\p@ \@minus5\p@
%    \abovedisplayshortskip \z@ \@plus3\p@
%    \belowdisplayshortskip 6\p@ \@plus3\p@ \@minus3\p@
%    \belowdisplayskip \abovedisplayskip
%    \let\@listi\@listI}
% \makeatother
\makeatletter
\renewcommand\normalsize{%
   \@setfontsize\normalsize\@xpt\@xiipt
   \abovedisplayskip 6\p@ \@plus2\p@ \@minus2\p@
   \abovedisplayshortskip \z@ \@plus3\p@
   \belowdisplayshortskip 6\p@ \@plus2\p@ \@minus2\p@
   \belowdisplayskip \abovedisplayskip
   \let\@listi\@listI}
\makeatother


%%%%%%%%%%%%%%%%%%%%%%%%%%%%%%%%%%%%%%%%%%%%%%%%%%%%%%%%%%%%%%%%%%%%%%
% 字体字号定义
%%%%%%%%%%%%%%%%%%%%%%%%%%%%%%%%%%%%%%%%%%%%%%%%%%%%%%%%%%%%%%%%%%%%%%
% 字号
\newcommand{\yihao}{\fontsize{26pt}{26pt}\selectfont}       % 一号
\newcommand{\xiaoyi}{\fontsize{24pt}{24pt}\selectfont}      % 小一
\newcommand{\erhao}{\fontsize{22pt}{22pt}\selectfont}       % 二号
\newcommand{\xiaoer}{\fontsize{18pt}{20pt}\selectfont}      % 小二
\newcommand{\sanhao}{\fontsize{16pt}{20pt}\selectfont}      % 三号   %固定行距,标题
\newcommand{\xiaosan}{\fontsize{15pt}{20pt}\selectfont}     % 小三
\newcommand{\sihao}{\fontsize{14pt}{20pt}\selectfont}       % 四号   %固定行距,标题
\newcommand{\xiaosi}{\fontsize{12pt}{20pt}\selectfont}      % 小四   %固定行距,标题,正文
\newcommand{\wuhao}{\fontsize{10.5pt}{17.5pt}\selectfont}   % 五号
\newcommand{\wuhaodan}{\fontsize{10.5pt}{10.5pt}\selectfont}% 五号,单倍行距 
\newcommand{\xiaowu}{\fontsize{9.5pt}{15pt}\selectfont}     % 小五
\newcommand{\xiaowudan}{\fontsize{9.5pt}{9.5pt}\selectfont} % 小五,单倍行距

% \newcommand{\yihao}{\fontsize{26pt}{39pt}\selectfont}	    % 一号,1.5  倍行距
% \newcommand{\xiaoyi}{\fontsize{24pt}{30pt}\selectfont}      % 小一,1.25 倍行距
% \newcommand{\erhao}{\fontsize{22pt}{27.5pt}\selectfont}     % 二号,1.25 倍行距
% \newcommand{\xiaoer}{\fontsize{18pt}{22.5pt}\selectfont}    % 小二,1.25 倍行距
% \newcommand{\sanhao}{\fontsize{16pt}{20pt}\selectfont}      % 三号,1.25 倍行距
% \newcommand{\xiaosan}{\fontsize{15pt}{19pt}\selectfont}     % 小三,1.25 倍行距
% \newcommand{\sihao}{\fontsize{14pt}{17.5pt}\selectfont}     % 四号,1.25倍行距
\newcommand{\daxiaosi}{\fontsize{12pt}{18pt}\selectfont}    % 小四,1.5 倍行距
% \newcommand{\xiaosi}{\fontsize{12pt}{15pt}\selectfont}      % 小四,1.25倍行距
\newcommand{\dawu}{\fontsize{10.5pt}{18pt}\selectfont}      % 五号,1.75倍行距
\newcommand{\zhongwu}{\fontsize{10.5pt}{16pt}\selectfont}   % 五号,1.5 倍行距
\newcommand{\wuhaogd}{\fontsize{10.5pt}{20pt}\selectfont}   % 五号,固定20pt行距,用于表、题
% \newcommand{\wuhao}{\fontsize{10.5pt}{10.5pt}\selectfont}   % 五号,单倍行距
% \newcommand{\xiaowu}{\fontsize{9pt}{9pt}\selectfont}	    % 小五,单倍行距

\newcommand{\song}{\CJKfamily{song}}
\newcommand{\hei}{\CJKfamily{hei}}
\newcommand{\kai}{\CJKfamily{kai}}
\newcommand{\fs}{\CJKfamily{fs}}
\newcommand{\xkai}{\CJKfamily{xkai}}

% defaultfont 默认字体命令
\def\defaultfont{\renewcommand{\baselinestretch}{1.27}
  \song\fontsize{12pt}{15pt}\selectfont}
% \def\defaultfont{%\renewcommand{\baselinestretch}{1}
%   \song\xiaosi}
% defaultfont 默认字体命令
% \def\defaultfont{\renewcommand{\baselinestretch}{1}%{1.27}
%   \song\fontsize{12pt}{20pt}\selectfont}

% 设置目录字体和行间距
\def\defaultmenufont{\song\fontsize{12pt}{20pt}\selectfont\setlength{\baselineskip}{20pt}}

% 固定距离内容填入及下划线
\makeatletter
\newcommand\fixeddistanceleft[2][1cm]{{\hb@xt@ #1{#2\hss}}}
\newcommand\fixeddistancecenter[2][1cm]{{\hb@xt@ #1{\hss#2\hss}}}
\newcommand\fixeddistanceright[2][1cm]{{\hb@xt@ #1{\hss#2}}}
\newcommand\fixedunderlineleft[2][1cm]{\underline{\hb@xt@ #1{#2\hss}}}
\newcommand\fixedunderlinecenter[2][1cm]{\underline{\hb@xt@ #1{\hss#2\hss}}}
\newcommand\fixedunderlineright[2][1cm]{\underline{\hb@xt@ #1{\hss#2}}}
\makeatother

%%%%%%%%%%%%%%%%%%%%%%%%%%%%%%%%%%%%%%%%%%%%%%%%%%%%%%%%%%%%%%%%%%%%%%
% 标题环境相关
%%%%%%%%%%%%%%%%%%%%%%%%%%%%%%%%%%%%%%%%%%%%%%%%%%%%%%%%%%%%%%%%%%%%%%
% 定义、定理等环境
\theoremstyle{plain}
\theoremheaderfont{\hei\boldmath}
\theorembodyfont{\song\rmfamily}
% \setlength{\theorempreskipamount}{5pt}
% \setlength{\theorempostskipamount}{-10pt}
\newtheorem{definition}{\hei 定义}[chapter]
\newtheorem{example}{\hei 例}[chapter]
\newtheorem{algorithm}{\hei 算法}[chapter]
\newtheorem{theorem}{\hei \qquad 定理}%[chapter] %%预答辩修改,全本编号?
\newtheorem{axiom}{\hei 公理}[chapter]
\newtheorem{proposition}[theorem]{\hei 命题}
\newtheorem{property}{\hei 性质}
\newtheorem{lemma}[theorem]{\hei 引理}
\newtheorem{corollary}{\hei 推论}[chapter]
\newtheorem{remark}{\hei 注解}[chapter]
\newenvironment{proof}{\hei{测试目的} }{\hfill $\square$ \vskip 4mm}
\newtheorem{method}{\hei 方法}[chapter]

% 定义标题和目录中章标题
\renewcommand\contentsname{目~~录}
\renewcommand\bibname{参考文献}
\newcommand\prechaptername{}   %%% 按照doc模板似乎不需要,但是看大家的论文都是这样的,保留!
\newcommand\postchaptername{}
\renewcommand\appendixname{附录}
\renewcommand\chaptername{\prechaptername\CJKnumber{\thechapter}\postchaptername}    %%%  这三行待添加
\newcommand\contentchaptername{\prechaptername\CJKnumber{\thecontentslabel}\postchaptername}


% 自定义项目列表标签及格式 \begin{publist} 列表项 \end{publist}
\newcounter{pubctr} %自定义新计数器
\newenvironment{publist}{%%%%%定义新环境
\begin{list}{[\arabic{pubctr}]} %%标签格式
    {
     \usecounter{pubctr}
     \setlength{\leftmargin}{2.5em}     % 左边界 \leftmargin =\itemindent + \labelwidth + \labelsep
     \setlength{\itemindent}{0em}     % 标号缩进量
     \setlength{\labelsep}{1em}       % 标号和列表项之间的距离,默认0.5em
     \setlength{\rightmargin}{0em}    % 右边界
     \setlength{\topsep}{0ex}         % 列表到上下文的垂直距离
     \setlength{\parsep}{0ex}         % 段落间距
     \setlength{\itemsep}{0ex}        % 标签间距
     \setlength{\listparindent}{0pt} % 段落缩进量
    }}
{\end{list}}%%%%%


%%%%%%%%%%%%%%%%%%%%%%%%%%%%%%%%%%%%%%%%%%%%%%%%%%%%%%%%%%%%%%%%%%%%%%
% 段落章节相关
%%%%%%%%%%%%%%%%%%%%%%%%%%%%%%%%%%%%%%%%%%%%%%%%%%%%%%%%%%%%%%%%%%%%%%
\setcounter{secnumdepth}{4} %
\setcounter{tocdepth}{3}    % 显示3级标题
% \setcounter{secnumdepth}{4}
% \setcounter{tocdepth}{4}


% 段落之间的竖直距离
\setlength{\parskip}{1.2pt}
% 段落缩进
\setlength{\parindent}{24pt}
% 定义行距
\renewcommand{\baselinestretch}{1.27}
% 参考文献条目间行间距
\setlength{\bibsep}{2pt}
% \setlength{\bibsep}{1.2pt}

%%%
% titlesec 宏包(章节标题设置)
% hang 标题头与标题内容在同一行; block 适合居中的标题;display 标题头与标题内容在两行上
% 四个大括号内依次为:标题的排版格式、标题头的定义、标题头与标题内容之间的距离、排版标题前执行的命令

% \titleformat{\chapter}[block]{\centering\sanhao\hei}{\chaptername}{1em}{} 
\titleformat{\chapter}[block]{\sanhao\hei}{\chaptername}{1em}{} 
\titleformat{\section}[hang]{\sihao\hei}{\thesection}{0.5em}{}
\titleformat{\subsection}[hang]{\xiaosi\hei}{\thesubsection}{0.5em}{}
\titleformat{\subsubsection}[hang]{\xiaosi\song}{\thesubsubsection}{0.5em}{}

% % titlesec 宏包(章节标题前后间距设置)
% %\titlespacing{命令}{左距离}{上距离}{下距离}
% % 段后间距设置: 大标题  30-36 pt    一级标题  18-24 pt    二级标题  12-15 pt
% \titlespacing{\chapter}{0pt}{0pt}{32pt plus 4pt minus 2pt}
% \titlespacing{\section}{0pt}{5pt}{15pt plus 4pt minus 2pt}
% \titlespacing{\subsection}{0pt}{3pt}{10pt plus 2pt minus 1pt}
% \titlespacing{\subsubsection}{0pt}{3pt}{9pt plus 1pt minus 1pt}

% titlesec 宏包(章节标题前后间距设置)
%\titlespacing{命令}{左距离}{上距离}{下距离}
% 段后间距设置: 大标题  30-36 pt    一级标题  18-24 pt    二级标题  12-15 pt
\titlespacing{\chapter}{0pt}{-14pt}{12pt}
\titlespacing{\section}{0pt}{7pt}{0pt}
\titlespacing{\subsection}{0pt}{6pt}{0pt}
\titlespacing{\subsubsection}{0pt}{3pt}{0pt}

% titlesec 宏包(中文目录缩进设置)  
\contentsmargin{3.0mm}    % 控制目录线同右边页码间空白距离


% % 中文目录中采用垂直居中的实心($\cdot$)圆点作为填充符号,所以需要重新定义章节目录项格式
% \titlecontents{chapter}[0.0em]{}{\contentchaptername}{}{{\titlerule*[5pt]{.}}\contentspage}
% \titlecontents{section}[1.0em]{}{\thecontentslabel}{}{{\titlerule*[5pt]{.}}\contentspage}
% \titlecontents{subsection}[2.0em]{}{\thecontentslabel}{}{{\titlerule*[5pt]{.}}\contentspage}
% \titlecontents{subsubsection}[3.0em]{}{\thecontentslabel}{}{{\titlerule*[5pt]{.}}\contentspage}

% \def\confont{\song\xiaosi}
% \titlecontents{chapter}[0.0em]{\confont}{\contentchaptername\hspace{0.5em}}{}{\hspace{0.5em}{\titlerule*[5pt]{.}}\contentspage}
% \titlecontents{section}[1.0em]{\confont}{\thecontentslabel\hspace{0.5em}}{}{\hspace{0.5em}{\titlerule*[5pt]{.}}\contentspage}
% \titlecontents{subsection}[2.0em]{\confont}{\thecontentslabel\hspace{0.5em}}{}{\hspace{0.5em}{\titlerule*[5pt]{.}}\contentspage}
% \titlecontents{subsubsection}[3.0em]{\confont}{\thecontentslabel\hspace{0.5em}}{}{\hspace{0.5em}{\titlerule*[5pt]{.}}\contentspage}

\titlecontents{chapter}[0.0em]{}{\contentchaptername\hspace{0.5em}}{}{\hspace{0.5em}{\titlerule*[5pt]{.}}\contentspage}
\titlecontents{section}[1.0em]{}{\thecontentslabel\hspace{0.5em}}{}{\hspace{0.5em}{\titlerule*[5pt]{.}}\contentspage}
\titlecontents{subsection}[2.0em]{}{\thecontentslabel\hspace{0.5em}}{}{\hspace{0.5em}{\titlerule*[5pt]{.}}\contentspage}
\titlecontents{subsubsection}[3.0em]{}{\thecontentslabel\hspace{0.5em}}{}{\hspace{0.5em}{\titlerule*[5pt]{.}}\contentspage}
%%%%%%%%%%%%%%%%%%%%%%%%%%%%%%%%%%%%%%%%%%%%%%%%%%%%%%%%%%%%%%%%%%%%%%
% 页眉页脚设置
%%%%%%%%%%%%%%%%%%%%%%%%%%%%%%%%%%%%%%%%%%%%%%%%%%%%%%%%%%%%%%%%%%%%%%

% \newcommand{\makeheadrule}{%
%   \makebox[0pt][l]{\rule[.7\baselineskip]{\headwidth}{0.5pt}}%
%   \vskip-.8\baselineskip}
% \newcommand{\makeheadrule}{%
%     \makebox[0pt][l]{\rule[.7\baselineskip]{\headwidth}{0.8pt}}%
%     \rule[0.85\baselineskip]{\headwidth}{1.5pt}\vskip-.8\baselineskip}%1.5 0.4-&gt;0.5
\newcommand{\makeheadrule}{%
    \hrule width\headwidth height2.8pt \vspace{1pt}%
    \hrule width\headwidth height0.8pt}

\makeatletter
\renewcommand{\headrule}{%
  {\if@fancyplain\let\headrulewidth\plainheadrulewidth\fi
    \makeheadrule}}


% % % 页眉 页脚
% % \pagestyle{fancy}
\pagestyle{fancyplain}
\fancyhf{}
%下面这行在页眉显示当前章节名称。 参考http://blog.sina.com.cn/s/blog_5e16f1770100rgux.html
% \renewcommand{\chaptermark}[1]{\markboth{\chaptername\  #1}{}}  
\renewcommand{\chaptermark}[1]{\markboth{%    %自己摸索的。#1应该表示章节名称,放在判断的最外面。
    \if@mainmatter
      \ifnum\arabic{chapter}>0 %
           \chaptername\quad
      \fi%
    \fi#1}{}} 
\fancyhead[CO]{\song\xiaowudan\leftmark}
\fancyhead[CE]{\song\xiaowudan\rightmark}
% \fancyhead[CE]{\song\xiaowudan{个人学习总结}}
\fancyfoot[C,C]{\song\wuhaodan$ $~\thepage~$ $}

% % Clear Header Style on the Last Empty Odd pages
% \makeatletter
% \def\cleardoublepage{\clearpage\if@twoside \ifodd\c@page\else%
%   \hbox{}%
%   \thispagestyle{empty}%              % Empty header styles
%   \newpage%
%   \if@twocolumn\hbox{}\newpage\fi\fi\fi}

% 中文封面
\def\titleC#1#2{\def\@titleC{#1}\def\@titleCN{#2}}
\def\@titleC{}
\def\@titleCN{}
\def\authorC#1{\def\@authorC{#1}}\def\@authorC{}
\def\majorC#1{\def\@majorC{#1}}\def\@majorC{}
\def\supervisorC#1{\def\@supervisorC{#1}}\def\@supervisorC{}
\def\dateC#1{\def\@dateC{#1}}\def\@dateC{}

\def\classno#1{\def\@classno{#1}}\def\@classno{}
\def\secretlevel#1{\def\@secretlevel{#1}}\def\@secretlevel{}
\def\authorno#1{\def\@authorno{#1}}\def\@authorno{}


% 英文封面
\def\titleE#1{\def\@titleE{#1}}\def\@titleE{}
\def\authorE#1{\def\@authorE{#1}}\def\@authorE{}
\def\majorE#1{\def\@majorE{#1}}\def\@majorE{}
\def\supervisorE#1{\def\@supervisorE{#1}}\def\@supervisorE{}
\def\dateE#1{\def\@dateE{#1}}\def\@dateE{}

%%%%%%%%%%%%%%%%%%%%%%%%%%%%%%%%%%%%%%%%%%%%%%%%%%%%%%%%%%%%%%%%%%%%%%
% 列表环境设置

%%%%%%%%%%%%%%%%%%%%%%%%%%%%%%%%%%%%%%%%%%%%%%%%%%%%%%%%%%%%%%%%%%%%%%

% \setlist[enumerate]{1、,itemsep=-5pt,topsep=0mm,labelindent=\parindent,leftmargin=*}
\setlist[enumerate]{(1),itemsep=-5pt,topsep=0mm,labelindent=\parindent,leftmargin=*}

%%%%%%%%%%%%%%%%%%%%%%%%%%%%%%%%%%%%%%%%%%%%%%%%%%%%%%%%%%%%%%%%%%%%%%
% 国际单位,以点连接。
%%%%%%%%%%%%%%%%%%%%%%%%%%%%%%%%%%%%%%%%%%%%%%%%%%%%%%%%%%%%%%%%%%%%%%
% \sisetup{inter-unit-product = { }\cdot{ }}

%%%%%%%%%%%%%%%%%%%%%%%%%%%%%%%%%%%%%%%%%%%%%%%%%%%%%%%%%%%%%%%%%%%%%%
% 参考文献的处理
%%%%%%%%%%%%%%%%%%%%%%%%%%%%%%%%%%%%%%%%%%%%%%%%%%%%%%%%%%%%%%%%%%%%%%

% \addtolength{\bibsep}{-0.5em}              % 缩小参考文献间的垂直间距
\setlength{\bibhang}{2em}
\bibpunct{[}{]}{,}{s}{}{}

%%%%%%%%%%%%%%%%%%%%%%%%%%%%%%%%%%%%%%%%%%%%%%%%%%%%%%%%%%%%%%%%%%%%%%
%   其他设置
%%%%%%%%%%%%%%%%%%%%%%%%%%%%%%%%%%%%%%%%%%%%%%%%%%%%%%%%%%%%%%%%%%%%%%
%   增加 \ucite 命令使显示的引用为上标形式
%   \newcommand{\ucite}[1]{$^{\mbox{\scriptsize \cite{#1}}}$}
\newcommand{\ucite}[1]{\scalebox{1.3}[1.3]{\raisebox{-0.65ex}{\cite{#1}}}}
%%%%%%%%%%%%%%%%%%%%%%%%%%%%%%%%%%%%%%%%%%%%%%%%%%%%%%%%%%%%%%%%%%%%%%
%   图形表格
%%%%%%%%%%%%%%%%%%%%%%%%%%%%%%%%%%%%%%%%%%%%%%%%%%%%%%%%%%%%%%%%%%%%%%
\renewcommand{\figurename}{图}
\renewcommand{\tablename}{表}
% \renewcommand{\thetable}{\thechapter-\arabic{table}} %修改表格的表头为表2-1
% \renewcommand{\thefigure}{\thechapter-\arabic{figure}} %修改图片的题头为图2-1
\@removefromreset{table}{chapter}
\@removefromreset{figure}{chapter}
\renewcommand{\thetable}{\arabic{table}} %修改表格的表头为表1
\renewcommand{\thefigure}{\arabic{figure}} %修改图片的题头为图1

% \renewcommand\listoftables{\vskip0.5cm
%      %\par{\Large\bfseries \listtablename}\vskip1cm
%      \section*{\listtablename}\vskip0.5cm
%        \@mkboth{%
%            \MakeUppercase\listtablename}%
%           {\MakeUppercase\listtablename}%
%      \@starttoc{lot}%
%      }

% \newcommand*{\noaddvspace}{\renewcommand*{\addvspace}[1]{}}
% \addtocontents{lof}{\protect\noaddvspace}
% \addtocontents{lot}{\protect\noaddvspace}


% \captionstyle{\centering}
% \hangcaption
\captiondelim{\hspace{1em}}
\captiondelim{\hspace{1em}}
\captionnamefont{\song \wuhaogd}%字体改为宋体,五号,20pt
\captiontitlefont{\song \wuhaogd}
\setlength{\abovecaptionskip}{5pt}
\setlength{\belowcaptionskip}{5pt}

\heavyrulewidth=2.25pt  %三线表中粗线磅值
\lightrulewidth=1pt   %细线磅值
\belowbottomsep=-10pt

\newcommand{\figskip}{\setlength{\abovecaptionskip}{0pt}\setlength{\belowcaptionskip}{-10pt}}

\newcommand{\tablepage}[2]{\begin{minipage}{#1}\vspace{0.5ex} #2 \vspace{0.5ex}\end{minipage}}
\newcommand{\returnpage}[2]{\begin{minipage}{#1}\vspace{0.5ex} #2 \vspace{-1.5ex}\end{minipage}}
%%%%%%%%%%%%%%%%%%%%%%%%%%%%%%%%%%%%%%%%%%%%%%%%%%%%%%%%%%%%%%%%%%%%%%
% 定义题头格言的格式
%%%%%%%%%%%%%%%%%%%%%%%%%%%%%%%%%%%%%%%%%%%%%%%%%%%%%%%%%%%%%%%%%%%%%%

\newsavebox{\AphorismAuthor}
\newenvironment{Aphorism}[1]
{\vspace{0.5cm}\begin{sloppypar} \slshape
    \sbox{\AphorismAuthor}{#1}
    \begin{quote}\small\itshape }
    {\\ \hspace*{\fill}------\hspace{0.2cm} \usebox{\AphorismAuthor}
    \end{quote}
  \end{sloppypar}\vspace{0.5cm}}

% 自定义一个空命令,用于注释掉文本中不需要的部分。
\newcommand{\comment}[1]{}

% This is the flag for longer version
\newcommand{\longer}[2]{#1}

\newcommand{\ds}{\displaystyle}

% define graph scale
\def\gs{1.0}

%%%%%%%%%%%%%%%%%%%%%%%%%%%%%%%%
%其他的设置

\newcommand{\hhl}[1]{\hl{\mbox{#1}}}  %定义一个用于中文高亮的语法

\renewcommand{\theequation}{\thechapter-\arabic{equation}} %修改图片的题头为图2-1

\newcommand{\til}{\textasciitilde}

%% 定义一组微分方程中的简便描述
% \newcommand{\dd}{\mathrm{d}}
\newcommand{\dd}{\mathop{}\!\mathrm{d}}
\newcommand{\qq}{\partial}
\newcommand{\tw}{\textwidth}
\def\Q#1#2{\frac{\partial#1}{\partial #2}}
\def\D#1#2{\frac{\dd#1}{\dd#2}}

\def\Ci#1{\setcitestyle{numbers}\!#1}  %非上标形式的引用

\newcommand{\uT}{\mathrm{T}} %用于上标
\renewcommand{\rm}{\mathrm}   %正体
\renewcommand{\bf}{\mathbf}   %正体加粗
\renewcommand{\le}{\leqslant} 
\renewcommand{\ge}{\geqslant} 

\newcounter{rowno}
\setcounter{rowno}{0}
\newcommand{\mynum}{{\arabic{rowno}\stepcounter{rowno}}}